% ----------------------------------------------------------------
% MANG2023 Effective Business Performance Assignment.tex
% ---------------------------------------------------------------- 
\documentclass{ecsarticle}     % Use the Article Style
%\documentclass[review]{elsarticle}
\usepackage{natbib}            % Use Natbib style for the refs.
\graphicspath{./Figures/}
\usepackage{appendix}
\usepackage{color}
\usepackage{ transparent}
\usepackage{lipsum}
\usepackage[colorinlistoftodos]{todonotes}
\usepackage{multirow}
\usepackage{wrapfig}
\newcommand{\inote}[1] {\todo[inline]{#1}}
\def\citeapos#1{\citeauthor{#1}'s (\citeyear{#1})}
\removecolourlinks    % Uncomment this command to remove colour from any links
% ----------------------------------------------------------------
\begin{document}
\frontmatter
\title      {MANG6143 Project Risk Management}
\authors    {\texorpdfstring
             {\href{mailto:tjs1g10@ecs.soton.ac.uk}{Thomas J. Smith}}
             {Thomas J.Smith}
            }
\purpose  {}
\company       {MEng Electronic Engineering with Power Systems}
\addresses  {\purname\\\coname\\\univname}
\date       {\today}
\subject    {Performance Uncertainty Management Processes - PUMPs}
\keywords   {}
\maketitle
\begin{abstract}
%Knowledge and Understanding
%On successful completion of the unit you will be able to:
%\begin{itemize}
%\item discuss the usefulness of a variety of risk management frameworks,
%\item explain the problems associated with measuring performance,
%\item explain the problems associated with estimating probabilities,
%\item explain related ‘qualitative’ issues,
%\item understand the motives for undertaking formal risk management processes,
%\item be aware of limitations of some common practice,
%\item describe the issues to be addressed in establishing a formal process.
%\end{itemize}
%Subject Specific Intellectual Skills
%On successful completion of the unit you will be able to:
%\begin{itemize}
%\item discuss the application of a formal risk management process in a project context,  
%\item show how to identify an effective structure for addressing uncertainty, opportunity and risk associated with any project.
%\end{itemize}
%
%Key Skills
%By the end of the unit you will develop your ability in the following skills: 
%numeracy in terms of uncertainty, opportunity and risk, group working and communication during the module and subsequently through the assignment, key skills in information handling, critical analysis and written communication.

\end{abstract}


% -----------------------
% lstpatch.sty
% -----------------------
% lstpatch cannot be distributed with these files. I believe it is only needed if the
% \lstlistoflistings is used. So this has been turned off by default. Re-add if required:
% \usepackage{lstpatch}
% \lstlistoflistings
% You will need to download lstpatch, possibly from:
% http://web.mit.edu/texsrc/source/latex/listings/lstpatch.sty
% -----------------------

\mainmatter
%  Pt1.tex
% !TeX spellcheck = en_GB
% !TeX root = ProjectRiskManagement.tex

\section{Approaches to Uncertainty and Underlying Complexity Management - 1200}

Introduction to risk and opportunity, underlying complexity. 

Challenge of traditional view of risk.

%Project life cycle introduction.
A traditional four stage view of the asset/change lifecycle is a useful starting point to consider the scope of a project.
The four stages are conceptualize, planning, execution and delivery (E\&D) and Utilization.
Effective uncertainty management requires a macro-view of the entire project context, so that corporate and operational uncertainty is captured in addition to planning uncertainty.
This leads to an elaboration of the lifecycle to incorporate 12 stages, each emphasizing a different management purpose and outcome.
Both views are shown in figure \ref{Figure:Project_Lifecycle}.

This paper is particularly concerned with the E\&D strategy shaping phase.
This is within the broader planning cqtegory, and follows the design,operation and termination strategy shaping phase.
The DOT phase aims are to ...
The E\



\begin{figure}[!h]
  \centering
    \includegraphics[width = \textwidth]{./Figures/ProjectLifecycleDetailedCurve.pdf} 
\caption{Twelve-stage asset/change lifecycle - adapted from \cite{chapman}}
\label{Figure:Project_Lifecycle}
\end{figure}

% execution and delivery strategy shaping phase of a project's life cycle
Procedures are a common way to ensure consistency and quality is maintained through a range of repeated applications.
While a good procedure is often designed to be simple, repeatable and transparent, this cannot be a uniform approach.
Some high complexity, high uncertainty projects require sophisticated, tailored procedures.
The PUMP framework supports this concept through the idea of PUMP packs, that is a set of PUMPs tailored to specific projects and project lifecycle stages.
This paper focusses on PUMPs within the context of the E\&D strategy shaping stage.


\subsection{The PUMP Process}
%Explain concisely in your own words what you believe are the key overall features of a PUMP approach to project risk management in the execution and delivery strategy shaping phase of a project’s lifecycle. Compare these features with the PMI PIMBOK approach or any other form of common practice you are familiar with if you find this helpful, but focus on the PUMP approach. Use examples to illustrate your discussion if you wish, but concentrate on concepts and principles. This will be a largely descriptive summary of your interpretation of the lectures and associated reading. It will demonstrate your grasp of the central core of the unit’s material as a whole, and should be approached with a view to demonstrating this understanding..



\subsection{Contrast with Other Risk Management Processes}
Contrast with PUMP and PMI PMBOK and others.

\subsection{The Clarity Efficient Approach}
Summarise that PUMPS offer a higher level of clarity for the project as a whole rather than 


1200words.
%  Pt2.tex
% !TeX spellcheck = en_GB
% !TeX root = ProjectRiskManagement.tex

\section{The Identify Phase - Identify all relevant sources of uncertainty 900} \label{s:Identify}
% the identify phase of the PUMP approach, in the execution and delivery strategy shaping phase of a project’s lifecycle, explain concisely in your own words what you believe are the key features of a PUMP approach, comparing these features with the PMI PIMBOK approach or any other form of common practice you are familiar with. Your discussion should demonstrate your ability to understand a particular area of the course material in depth, based on selective reading, critical analysis and the case study exercise. Use examples to illustrate your discussion if you wish, making use of the Samdo case study if you wish, but concentrate on concepts and principles. Build on your Part 1 answer, avoiding repetition of earlier discussion.
%900words.

%Key Features

%Compare with PMI PMBOK or others

%In depth understanding

The identify phase is commonplace in several risk management methodologies including PUMPs. 
However, there is a marked removal from best practice in the PUMP approach resulting in an iterative, inherently creative method to identify all relevant sources of uncertainty, possible response options, assumptions, conditions and second order sources.
This encourages the consideration of all types of uncertainty in a clarity efficient methodology.

The common practice approach is a linear process whereby all relevant risk events must be identified at an early stage. 
The response identification or formulation of mitigation strategies is left until later in the linear process. 
Assumptions and conditions are decoupled, since unbiased estimation is not a formal goal of the process. 
Common practice offers an inadequate level of clarity for effective management of uncertainty and fails to achieve an acceptable level of clarity efficiency.
Moreover, a dangerous false sense of security can be created through best practice approaches which may endanger the achievement of project objectives.
                         
Within the PUMP framework, the identification of sources of uncertainty and possible response options are closely coupled. 
Considering response options as new sources are identified increases the chances of developing powerful general responses that are crucial to deliver clarity efficiency. 
Moreover, unidentified responses contribute to ambiguity uncertainty. 
On some occasions the consequences of the response can lead to secondary sources which may require consideration.
The identify phase allows an enlightened reshaping of relevant base plans and contingency plans as required.

The process has two main features. 
The \textbf{search} task involves finding all relevant sources responses and conditions. 
The \textbf{classify} task provides a suitable structure where sources have been aggregated or decomposed as necessary upon which further analysis can proceed.
The process is shown in figure \ref{}.

\inote{Identify phase flowchart}

The identification of responses is 

\inote{Generic Response Types Figure}







%  Pt3.tex
% !TeX spellcheck = en_GB
% !TeX root = ProjectRiskManagement.tex
\section{Evaluating Risk Management Processes} \label{s:Evaluate}

For the evaluate phase of the PUMP approach, in the execution and delivery strategy shaping phase of a project’s lifecycle, explain concisely in your own words what you believe are the key features of a PUMP approach, comparing these features with the PMI PIMBOK approach or any other form of common practice you are familiar with. Your discussion should demonstrate your ability to understand a particular area of the course material in depth, based on selective reading, critical analysis and the case study exercise. Use examples to illustrate your discussion if you wish, and make use of the Transcon case study if you wish, but concentrate on concepts and principles. Build on your Parts 1 and 2 answers, avoiding repetition.

900 words.

\section{Concluding Remarks}
Risk management processes have evolved hugely in recent years.
Best practice approaches are increasingly harnessing the concepts of opportunity efficiency and clarity efficiency rather than rigid, linear, event-centric methodologies of the past \citep{association2012}.
However, the PUMP approach remains a forerunner to best practice in the focus on opportunity, an encompassing perspective of uncertainty and in the attention to tackling underlying complexity and assumptions.
This is achieved while increasing the clarity efficiency of the overall process.
Personally, I have discovered and understood concepts that have shifted my perspective from a risk event focus to a holistic appreciation of uncertainty, fundamentally altering my approach to future projects and enriching my career as an engineer.
\backmatter
\bibliographystyle{apalike}
\bibliography{ECS}


\appendix
\appendixpage
\appendixheaderon
%\input{Terms}
\section{Acknowledgements}
This work builds on the ideas in \cite{chapman}. The work is not referenced unless there is direct quotation or adaption from the text. However, it remains the main source for this report and it's influence is kindly acknowledged.


\end{document}